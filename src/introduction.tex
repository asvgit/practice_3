\section{Задание на практику}
	В ходе практики должны быть освоены компетенции:
		\begin{itemize}
			\item ...
		\end{itemize}

\newpage
\section{Введение}

	Развитие лифтостроения в значительной мере определяет возможности современного высотного строительства,
		так как предельная высота здания ограничивается, во-первых, пропускной способностью его вертикального транспорта,
		во - вторых, надежностью лифтовой шахты. Возможности лифтового оборудования зависят от достижений в области
		науки и техники, а также успехами экономического развития промышленно развитых стран.

	Развитие инфраструктуры современных мегаполисов невозможно без многоэтажных жилых, производственных и офисных зданий,
		в которых эксплуатируются несколько лифтов, объединенных в группы. В этой связи актуальным видится вопрос
		совершенствования взаимной работы таких групп с целью повышения их эффективности.
		Это позволит как уменьшить число лифтов группы, так и сократить энергетические затраты существующих,
		что особенно уместно на фоне постоянного роста тарифов на электроэнергию.
