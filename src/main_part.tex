\stepcounter{mysection}\section{\arabic{mysection} Управление группой лифтов}

	Считается, что наилучшим алгоритмом управления группой лифтов является тот, который обеспечивает минимальные значения
		времени ожидания кабины пассажирами на посадочном этаже и перемещения их между этажами, а также минимизирует расход
		электроэнергии и при этом не требует значительных финансовых затрат [1].
		При определении параметров группы лифтов и алгоритмов управления требуется аналитически оценить характеристики
		пассажиропотока здания.
		Основными такими характеристиками являются: интенсивность пассажиропотока, время ожидания кабины на этаже и время
		кругового рейса кабины [2].
		
	Сложностью при определении указанных параметров является факт, что загрузка лифтового
		оборудования зданий и сооружений изменяется во времени по случайному закону. Однако здания и сооружения схожего типа
		и назначения характеризуются общностью так называемых всплесков интенсивности использования лифтов.
		Например, в административных зданиях и учебных заведениях расписание трудовой и учебной деятельности предопределяет
		характерные всплески интенсивности пассажиропотоков в начале и конце рабочего дня.
		Напротив, в жилых домах массовой застройки утренние и вечерние всплески интенсивности пассажиропотоков
		менее чётко выражены.

	В любом случае, если лифтовая система обеспечить критический пассажиропоток, то она обеспечить и любой другой пассажиропоток
		в любое время суток. Критический поток определяется некоторым временем, когда количество пассажиров лифта, поступающих на
		основной посадочный этаж за единицу времени, максимально. Если максимальный пассажиропоток значителен, то это сути не меняет.
		Если пассажиропоток относительно постоянен в течение суток, задача существенно упрощается, так как лифтовая система
		призвана обеспечить именно этот постоянный по величине поток.

	В настоящее время применяются два метода оценки характеристик пассажиропотока здания: метод моделирования и метод калькуляции.
		Исходной информацией для обоих методов является [2, 4, 5, 6]:

	1. Характеристики здания: количество жильцов на верхних этажах (исходя из предположения об одинаковом среднем
		количестве жильцов на этаже); количество обслуживаемых этажей; расстояние, которое проходит лифт от основного посадочного
		до последнего этажа и от основного посадочного до первого верхнего этажа; требуемая пропускная способность лифта (группы лифтов)
		
	2. Характеристики лифта: грузоподъемность кабины; коэффициент заполнения кабины, скорость лифта: время срабатывания тормоза;
		задержка при пуске привода; величина ускорения/замедления кабины, рывок (темп изменения ускорения/замедления);
		тип дверей (открывающиеся в середине или телескопические): ширина дверей; время открытия/закрытия дверей;
		время блокировки дверей (время задержки перед закрытием дверей после того, как в них прошел последний пассажир);
		предварительное время открытия дверей (инициация открыли я дверей, прежде чем кабина достигла уровня этажа).

	3. Статистическая информация: средний вес пассажира (68-80 кг, зависит от действующих строительных нормативов);
		время входа.выхода пассажира и др.

	4. Особые факторы: этажи с ресторанами, банками, конференц-залами и т.п.
	 
	В случае оценки пассажиропотока для жилых зданий массовой застройки используется метод калькуляции.
		В отечественной литературе по лифтам приведены расчетные зависимости [3, 7], соответствующие, в свою очередь, методическим
		основам расчета пассажирского вертикального транспорта, изложенным в «Пособии по проектированию общественных зданий и сооружений»
		(приложение к СНиП 2.08.02) ГОСТ Р 52941-2008. В этом случае рассчитывается время кругового рейса кабины -
		время между двумя последовательными отравлениями вверх кабины одного того же лифта с основного посадочного этажа,
		которое включает в себя время на движение вверх до этажа назначения, поворота и вниз до основного посадочного этажа,
		а также время на остановки и стоянку на этажах.

	Применение метода моделирования целесообразно для анализа пассажиропотока высотных домов, где требуется индивидуальный
		логический подход к характеристикам лифтового оборудования для повышения комфортности поездок и снижения затрат энергопотребления.
		Отечественными и зарубежным учеными было проведено множество исследований посвященных как непосредственно анализу пассажиропотока зданий,
		так разработке методик его определения и моделирования [2, 4, 5, 8, 9]. При моделировании весь предполагаемый поток людей
		переносится в виртуальный. Это достигается путем генерации случайных чисел и использовании идентичных реальным алгоритмов
		управления группой лифтов.
		
	Результатом моделирования является набор данных, подлежащих статистической обработке, после чего осуществляется выбор системы (алгоритма)
		управления группой лифтов. Классический алгоритм основывается на принципах парных и групповых систем управления,
		которые подразумевают выполнение следующих задач [7]:

	1. На каком этаже здания воза каждого лифта устанавливается вызывная панель (пост), имеющая две кнопки вызова (вверх и вниз).

	2. Кабина каждого лифта содержит внутреннюю панель с количеством кнопок, равным числу обслуживаемых этажей.

	3. В случае нажатия кнопки вызова система управления ищет первый попавшийся свободный лифт обеспечивает его остановку на этаже вызова.
		В случае отсутствия свободной кабины система переходит в режим ожидания, а затем, при освобождении кабины,
		отправляет ее по соответствующему вызову.

	4.В случае если через этаж, на котором инициирован вызов, проходит попутный лифт, то попутный лифт останавливается системой управления на данном этаже.

	5. После нажатия пассажиром кнопки внутри лифта, система управления отправляет лифт на этаж назначения, ставя при этом высший приоритет
		выполнения на ближайший этаж в соответствии с текущим направлением движения лифта.
		При этом во время движения лифта учитывается логика использования попутного лифта, описанная в предыдущем пункте.

	Усовершенствованные системы управления группой лифтов основываются на классическом алгоритме, дополненным одним из так называемых конфликтных критериев
		в качестве целевой функции. Чаще всего, таким критерием выступает минимизация затрат электроэнергии. То есть, оптимальным считается рейс
		с наименьшим расходом электроэнергии и временем ожидания не более 90 с. Сокращение времени ожидания и количества потребляемой энергии -
		две противоположные задачи, поэтому при выборе алгоритма управления вводятся весовые коэффициенты.
		В настоящее время они устанавливаются проектировщиком и являются постоянными величинами. Выбор подходящих значений этих коэффициентов
		в каждой ситуации движения приводит либо к сокращению количества электроэнергии, необходимого для перевозки пассажиров за счет снижения скорости
		их обслуживания, либо к обеспечению пассажиров более быстрым сервисом за счет увеличения энергозатрат.

	Качественная работа лифтов и их надежность остается одним из ключевых аспектов в деле обеспечения безопасности жилых и общественных зданий,
		что обуславливает усложнение систем управления процессом передвижения. Поэтому ведущие производители лифтового оборудования осуществляют
		постоянную разработку новых все более эффективных систем управления лифтами, а также модернизацию уже существующих алгоритмов.

	На сегодняшний день не существует оптимального алгоритма управления системой группой лифтов, так как каждый из них имеет ряд недостатков.

	Например, в основе классического алгоритма заложен принцип парного и группового управления, который в свою очередь базируется на собирательном управлении,
		то есть вызовы должны регистрироваться и выполняться в соответствии с ограниченным числом условий, которые должны быть учтены при проектировании
		системы управления. Поскольку ограничено количество условий, по которым система управления распределяет вызовы между кабинами
		и свободные кабины по высоте здания, то любая система не во всех ситуациях действует наилучшим образом.
		При этом, чем больше условий учтено при проектировании системы группового управления, тем система сложнее,
		что в свою очередь приводит к увеличению числа полупроводниковых логических элементов.

	Недостатком усовершенствованной системы управления является тот факт, что, несмотря на ввод конфликтного критерия,
		система не может обеспечить одновременно комфортную доставку пассажиров без увеличения энергозатрат.

	Таким образом, актуальной является задача разработки алгоритма, в основе которого заложен принцип северного управления лифтовыми группами с целью реализации
		более сложных алгоритмов без увеличения числа дополнительных логических элементов. Северное управление также позволит внедрять большее число
		дополнительных функций и связывать между собой согласованную работу лифтов наряду с обеспечением комфортной доставки пассажиров без увеличения энергозатрат,
		что отсутствовало в существующих алгоритмах При разработке подобных систем в качестве показателя эффективности работы целесообразно использовать
		число внедряемых функций наряду с уже существующими показателями эффективности`

