\input{preamble.tex}
\usepackage{hyperref,xcolor}

\usepackage{tocloft}
\renewcommand{\thesection}{}
\renewcommand{\thesubsection}{\arabic{mysection}.\arabic{subsection}}

\providecommand\mytitle{HП.430200.090404.000.ПЗ}
\providecommand\doctitle{Производственная - научно-исследовательская работа в семестре}

\begin{document}
\input{title.tex}
\setcounter{page}{2}
\tableofcontents

\newpage
\section{Задание на практику}
	Провести анализ существующих решений в сфере управления группой лифтов. А также выявить основные принципы, сильные и слабые стороны текущих решений.

	В ходе практики должны быть освоены компетенции:
		\begin{itemize}
			\item способность совершенствовать и развивать свой интеллектуальный и общекультурный уровень;
			\item способность к самостоятельному обучению новым методам исследования, к изменению научного и научно-производственного профиля своей профессиональной деятельности;
			\item умение оформлять отчеты о проведенной научно-исследовательской работе и подготавливать публикации по результатам исследования.
		\end{itemize}

\newpage
\section{Введение}

	Развитие лифтостроения в значительной мере определяет возможности современного высотного строительства,
		так как предельная высота здания ограничивается, во-первых, пропускной способностью его вертикального транспорта,
		во - вторых, надежностью лифтовой шахты. Возможности лифтового оборудования зависят от достижений в области
		науки и техники, а также успехами экономического развития промышленно развитых стран.

	Развитие инфраструктуры современных мегаполисов невозможно без многоэтажных жилых, производственных и офисных зданий,
		в которых эксплуатируются несколько лифтов, объединенных в группы. В этой связи актуальным видится вопрос
		совершенствования взаимной работы таких групп с целью повышения их эффективности.
		Это позволит как уменьшить число лифтов группы, так и сократить энергетические затраты существующих,
		что особенно уместно на фоне постоянного роста тарифов на электроэнергию.

\input{src/project.tex}
\newpage
\section{Заключение}
	В результате прохождения практики был выполнен обзор текущих решений в сфере управления группой лифтов.

	А также получены практические навык воспринимать математические, естественнонаучные, социально-экономические и профессиональные знания, умением самостоятельно приобретать, развивать и применять их для решения нестандартных задач, в том числе в новой или незнакомой среде и в междисциплинарном контексте.


\newpage
\section{Литература}
1. Программное обеспечение моделирования работы группы лифтов
/\ А.П. Шибанов [и др.] \/\/ Промышленные АСУ и контроллеры. 2003. № 8.
С. 42-43.

2. Финч Б Регулировка компьютеризироваииых систем управления
работой лифтов \/\/ Лифт. 2005. №1 [8]. С. 13-29.

3. Лифты. Учебник для вузов \/ Под ред. В.Д. Волкова М,: Изд-во
АСВ. 1999. 480 с.

4. Питерс Р., Мехта П, Схемы нагрузки пассажирских лифтов \/\/
Лифт. 2004, №2 [1]. С. 20-37.

5. Фочн Д.У. Выбор вызовов из холла по заданному месту назначе-
ния для лифтов с двойной платформой (трехмерное кодирование) \/\/ Лифт.
2005. №11 [18]. С. 4-10.

6. Анцсв В.Ю., Витчук П.В., Плахова Е‚А‚ Взаимосвязь характери—
стик пассажиропотока здания и износа лифтовых канатоведущих шкивов \/\/
Механика и физика процессов на поверхности и в контакте твердых тел,
деталей технологического и энергетического оборудования: межвуз. сб.
науч‹ тр.; под ред. В.В. Измайлова. Вып. \& Тверь: ТвГГУ, 2013. С. 110-114.

7. Егоров КА. Системы управления пассажирскими лифтами, м:
Стройиздаъ 19774 236 с.

8. Чжаиру В., Хонгши Л., Хуачжун Ю. Система текущего контроля
состояния лифта в реальном врсмсни \/\/ Лифт. 2005. №4 [11]. С. 12-16.

9. Шаныто А.А., Наумов Л.А. Искусство программирования лифта.
Объектно—ориентированное программирование с явным выделением со-
стояний \/\/ Информационно-управляющие системы. 2003. № 6. С. 38-49.
\end{document}
