\documentclass[a4paper,14pt]{extarticle}
\usepackage[utf8]{inputenc}
\usepackage[english,russian]{babel}
\usepackage{indentfirst}
\usepackage{misccorr}
\usepackage{graphicx}
\usepackage{amsmath}
\usepackage{luatextra}
\setmainfont{Times New Roman}
\setmonofont{Courier New}
\usepackage[left=3cm,right=1.5cm,
    top=1.5cm,bottom=2cm,bindingoffset=0cm]{geometry}
\providecommand\No{\char"2116}%{№}%
% \usepackage{sectsty}
% \changefontsizes[21pt]{14pt}
% \sectionfont{\fontsize{14}{14}\selectfont}

\usepackage{titlesec}

\usepackage{setspace}
\newcounter{mysection}
\titleformat{\section}
  {\centering\normalfont\fontsize{16}{16}\bfseries}{}{1em}{}
\titleformat{\subsection}
  {\normalfont\fontsize{14}{14}\bfseries}{\arabic{mysection}.\arabic{subsection}}{1em}{}

\usepackage{tikz}
\usepackage{verbatim}
\usepackage{pgfplots}
\usetikzlibrary{chains, shapes.misc}
\usetikzlibrary{shapes,arrows}
\usetikzlibrary{shapes, arrows, chains}

\tikzset{
	line/.style={draw, -latex'},
	every join/.style={line},
	u/.style={anchor=south},
	r/.style={anchor=west},
	fxd/.style={text width = 6em},
	it/.style={font={\small\itshape}},
	bf/.style={font={\small\bfseries}}
}
\tikzstyle{base} =
	[
		draw,
		on chain,
		on grid,
		align=center,
		minimum height=4ex,
		minimum width = 10ex,
		node distance = 6mm and 60mm,
		text badly centered
	]
\tikzstyle{coord} =
	[
		coordinate,
		on chain,
		on grid
	]
\tikzstyle{cloud} =
	[
		base,
		ellipse,
		fill = red!5,
		node distance = 3cm,
		minimum height = 2em
	]
\tikzstyle{decision} =
	[
		base,
		diamond,
		aspect=2,
		fill = green!10,
		node distance = 2cm,
		inner sep = 0pt
	]
\tikzstyle{block} =
	[
		rectangle,
		base,
		fill = blue!3,
		rounded corners,
		minimum height = 2em
	]
\tikzstyle{print_block} =
	[
		base,
		tape,
		tape bend top=none,
		fill = yellow!10
	]
\tikzstyle{io} =
	[
		base,
		trapezium,
		trapezium left angle = 70,
		trapezium right angle = 110,
		fill = blue!5
	]
\tikzstyle{for} = 
	[
		base,
		shape=chamfered rectangle,
		chamfered rectangle xsep=2cm, 
	]
\makeatletter
\pgfkeys{/pgf/.cd,
	subrtshape w/.initial=2mm,
	cycleshape w/.initial=2mm
}
\pgfdeclareshape{subrtshape}{
	\inheritsavedanchors[from=rectangle]
	\inheritanchorborder[from=rectangle]
	\inheritanchor[from=rectangle]{north}
	\inheritanchor[from=rectangle]{center}
	\inheritanchor[from=rectangle]{west}
	\inheritanchor[from=rectangle]{east}
	\inheritanchor[from=rectangle]{mid}
	\inheritanchor[from=rectangle]{base}
	\inheritanchor[from=rectangle]{south}
	\backgroundpath{
		\southwest \pgf@xa=\pgf@x \pgf@ya=\pgf@y
		\northeast \pgf@xb=\pgf@x \pgf@yb=\pgf@y
		\pgfmathsetlength\pgfutil@tempdima{\pgfkeysvalueof{/pgf/subrtshape w}}
		\def\ppd@offset{\pgfpoint{\pgfutil@tempdima}{0ex}}
		\def\ppd@offsetm{\pgfpoint{-\pgfutil@tempdima}{0ex}}
		\pgfpathmoveto{\pgfqpoint{\pgf@xa}{\pgf@ya}}
		\pgfpathlineto{\pgfqpoint{\pgf@xb}{\pgf@ya}}
		\pgfpathlineto{\pgfqpoint{\pgf@xb}{\pgf@yb}}
		\pgfpathlineto{\pgfqpoint{\pgf@xa}{\pgf@yb}}
		\pgfpathclose
		\pgfpathmoveto{\pgfpointadd{\pgfpoint{\pgf@xa}{\pgf@yb}}{\ppd@offsetm}}
		\pgfpathlineto{\pgfpointadd{\pgfpoint{\pgf@xa}{\pgf@ya}}{\ppd@offsetm}}
		\pgfpathlineto{\pgfpointadd{\pgfpoint{\pgf@xb}{\pgf@ya}}{\ppd@offset}}
		\pgfpathlineto{\pgfpointadd{\pgfpoint{\pgf@xb}{\pgf@yb}}{\ppd@offset}}
		\pgfpathclose
	}
}
\pgfdeclareshape{cyclebegshape}{
	\inheritsavedanchors[from=rectangle]
	\inheritanchorborder[from=rectangle]
	\inheritanchor[from=rectangle]{north}
	\inheritanchor[from=rectangle]{center}
	\inheritanchor[from=rectangle]{west}
	\inheritanchor[from=rectangle]{east}
	\inheritanchor[from=rectangle]{mid}
	\inheritanchor[from=rectangle]{base}
	\inheritanchor[from=rectangle]{south}
	\backgroundpath{
		\southwest \pgf@xa=\pgf@x \pgf@ya=\pgf@y
		\northeast \pgf@xb=\pgf@x \pgf@yb=\pgf@y
		\pgfmathsetlength\pgfutil@tempdima{\pgfkeysvalueof{/pgf/cycleshape w}}
		\pgfpathmoveto{\pgfqpoint{\pgf@xa}{\pgf@ya}}
\pgfpathlineto{\pgfpointadd{\pgfpoint{\pgf@xa}{\pgf@yb}}{\pgfpoint{0ex}{-\pgfutil@tempdima}}}
\pgfpathlineto{\pgfpointadd{\pgfpoint{\pgf@xa}{\pgf@yb}}{\pgfpoint{\pgfutil@tempdima}{0ex}}}
\pgfpathlineto{\pgfpointadd{\pgfpoint{\pgf@xb}{\pgf@yb}}{\pgfpoint{-\pgfutil@tempdima}{0ex}}}
\pgfpathlineto{\pgfpointadd{\pgfpoint{\pgf@xb}{\pgf@yb}}{\pgfpoint{0ex}{-\pgfutil@tempdima}}}
\pgfpathlineto{\pgfqpoint{\pgf@xb}{\pgf@ya}}
		\pgfpathclose
	}
}
\pgfdeclareshape{cycleendshape}{
	\inheritsavedanchors[from=rectangle]
	\inheritanchorborder[from=rectangle]
	\inheritanchor[from=rectangle]{north}
	\inheritanchor[from=rectangle]{center}
	\inheritanchor[from=rectangle]{west}
	\inheritanchor[from=rectangle]{east}
	\inheritanchor[from=rectangle]{mid}
	\inheritanchor[from=rectangle]{base}
	\inheritanchor[from=rectangle]{south}
	\backgroundpath{
		\southwest \pgf@xa=\pgf@x \pgf@ya=\pgf@y
		\northeast \pgf@xb=\pgf@x \pgf@yb=\pgf@y
		\pgfmathsetlength\pgfutil@tempdima{\pgfkeysvalueof{/pgf/cycleshape w}}
		\pgfpathmoveto{\pgfqpoint{\pgf@xb}{\pgf@yb}}
\pgfpathlineto{\pgfpointadd{\pgfpoint{\pgf@xb}{\pgf@ya}}{\pgfpoint{0ex}{\pgfutil@tempdima}}}
\pgfpathlineto{\pgfpointadd{\pgfpoint{\pgf@xb}{\pgf@ya}}{\pgfpoint{-\pgfutil@tempdima}{0ex}}}
\pgfpathlineto{\pgfpointadd{\pgfpoint{\pgf@xa}{\pgf@ya}}{\pgfpoint{\pgfutil@tempdima}{0ex}}}
\pgfpathlineto{\pgfpointadd{\pgfpoint{\pgf@xa}{\pgf@ya}}{\pgfpoint{0ex}{\pgfutil@tempdima}}}
\pgfpathlineto{\pgfqpoint{\pgf@xa}{\pgf@yb}}
		\pgfpathclose
	}
}
\makeatother
\tikzstyle{subroutine} =
	[
		base,
		subrtshape,
		fill = green!25
	]
\tikzstyle{cyclebegin} =
	[
		base,
		cyclebegshape,
		fill = blue!25
	]
\tikzstyle{cycleend} =
	[
		base,
		cycleendshape,
		fill = blue!25
	]
\tikzstyle{connector} =
	[
		base,
		circle,
		fill = red!25
	]

\usepackage{listings}
\usepackage{xcolor}
\lstset { %
    language=C++,
    backgroundcolor=\color{black!5}, % set backgroundcolor
    basicstyle=\footnotesize,% basic font setting
}
\lstdefinestyle{customc}{
  belowcaptionskip=1\baselineskip,
  breaklines=true,
  frame=L,
  xleftmargin=\parindent,
  language=C,
  showstringspaces=false,
  basicstyle=\footnotesize\ttfamily,
  keywordstyle=\bfseries\color{green!40!black},
  commentstyle=\itshape\color{purple!40!black},
  identifierstyle=\color{blue},
  stringstyle=\color{orange},
}

\lstdefinestyle{customasm}{
  belowcaptionskip=1\baselineskip,
  frame=L,
  xleftmargin=\parindent,
  language=[x86masm]Assembler,
  basicstyle=\footnotesize\ttfamily,
  commentstyle=\itshape\color{purple!40!black},
}

\lstset{escapechar=@,style=customc}

\usepackage{hyperref,xcolor}

\usepackage{tocloft}
\renewcommand{\thesection}{}
\renewcommand{\thesubsection}{\arabic{mysection}.\arabic{subsection}}

\providecommand\mytitle{HП.430200.090404.000.ПЗ}
\providecommand\doctitle{Производственная - научно-исследовательская работа в семестре}

\begin{document}
\begin{titlepage}
	\newpage
	\begin{center}
		ФЕДЕРАЛЬНОЕ АГЕНСТВО ЖЕЛЕЗНОДОРОЖНОГО ТАРНСПОРТА \\
		\vspace{14pt}
		Федерально государственное бюджетное образовательное учреждение \\\vspace{7pt} высшего образования \\\vspace{7pt}
		<<Иркутский государственный университет путей и сообщения>> \\\vspace{7pt}
		(ФГБОУ ВО ИрГУПС) \\\vspace{7pt}
		Факультет <<Управление на транспорте и информационнные технологии>> \\\vspace{7pt}
		Кафедра <<Информационные системы и защита информации>>
	\end{center}
	\vspace{42pt}
	\begin{center}
		ОТЧЕТ ПО ПРАКТИКЕ
	\end{center}
	\vspace{-14pt}
	\begin{center}
		\doctitle\\
	\vspace{14pt}
		\mytitle
	\end{center}
	\vspace{56pt}
	\begin{flushleft}
		\begin{tabular}{p{0.57\textwidth}l}
			Выполнил:
				&	Проверил: \\
			студент группы ПИм.1-16-1, Арляпов С.В.
				&	ст. пр. Звонков И.В.\\
			Шифр: 1621345
				&	\_\_\_\_\_\_\_\_\_\_\_\_\_\_\_  \\
			&\\
			&\\
			<<\_\_\_>>\_\_\_\_\_\_\_\_\_\_\_\_\_\_\_20\_\_г.
				&	<<\_\_\_>>\_\_\_\_\_\_\_\_\_\_\_\_\_\_\_20\_\_г.
		\end{tabular}
	\end{flushleft}
	\vspace{\fill}
	\begin{center}
		Иркутск 2017
	\end{center}
\end{titlepage}

\setcounter{page}{2}
\tableofcontents

\newpage
\section{Задание на практику}
	В ходе практики должны быть освоены компетенции:
		\begin{itemize}
			\item ...
		\end{itemize}

\newpage
\section{Введение}

	Развитие лифтостроения в значительной мере определяет возможности современного высотного строительства,
		так как предельная высота здания ограничивается, во-первых, пропускной способностью его вертикального транспорта,
		во - вторых, надежностью лифтовой шахты. Возможности лифтового оборудования зависят от достижений в области
		науки и техники, а также успехами экономического развития промышленно развитых стран.

	Развитие инфраструктуры современных мегаполисов невозможно без многоэтажных жилых, производственных и офисных зданий,
		в которых эксплуатируются несколько лифтов, объединенных в группы. В этой связи актуальным видится вопрос
		совершенствования взаимной работы таких групп с целью повышения их эффективности.
		Это позволит как уменьшить число лифтов группы, так и сократить энергетические затраты существующих,
		что особенно уместно на фоне постоянного роста тарифов на электроэнергию.

\newpage
\stepcounter{mysection}\section{\arabic{mysection} Теоретическая часть}

...

\newpage
\stepcounter{mysection}\section{\arabic{mysection} Основная часть}

	...


\newpage
\section{Заключение}
	В результате прохождения практики ...

	А также получены практические навыки в ...


\newpage
\section{Литература}
1. Программное обеспечение моделирования работы группы лифтов
/\ А.П. Шибанов [и др.] \/\/ Промышленные АСУ и контроллеры. 2003. № 8.
С. 42-43.

2. Финч Б Регулировка компьютеризироваииых систем управления
работой лифтов \/\/ Лифт. 2005. №1 [8]. С. 13-29.

3. Лифты. Учебник для вузов \/ Под ред. В.Д. Волкова М,: Изд-во
АСВ. 1999. 480 с.

4. Питерс Р., Мехта П, Схемы нагрузки пассажирских лифтов \/\/
Лифт. 2004, №2 [1]. С. 20-37.

5. Фочн Д.У. Выбор вызовов из холла по заданному месту назначе-
ния для лифтов с двойной платформой (трехмерное кодирование) \/\/ Лифт.
2005. №11 [18]. С. 4-10.

6. Анцсв В.Ю., Витчук П.В., Плахова Е‚А‚ Взаимосвязь характери—
стик пассажиропотока здания и износа лифтовых канатоведущих шкивов \/\/
Механика и физика процессов на поверхности и в контакте твердых тел,
деталей технологического и энергетического оборудования: межвуз. сб.
науч‹ тр.; под ред. В.В. Измайлова. Вып. \& Тверь: ТвГГУ, 2013. С. 110-114.

7. Егоров КА. Системы управления пассажирскими лифтами, м:
Стройиздаъ 19774 236 с.

8. Чжаиру В., Хонгши Л., Хуачжун Ю. Система текущего контроля
состояния лифта в реальном врсмсни \/\/ Лифт. 2005. №4 [11]. С. 12-16.

9. Шаныто А.А., Наумов Л.А. Искусство программирования лифта.
Объектно—ориентированное программирование с явным выделением со-
стояний \/\/ Информационно-управляющие системы. 2003. № 6. С. 38-49.
\end{document}
